\documentclass{tudscrexsht} % default math fonts
% \documentclass[german]{tudscrexsht} % German language
% \documentclass[sansmath]{tudscrexsht} % sans math fonts
% \documentclass[mathpazo]{tudscrexsht} % mathpazo math fonts

\usepackage[
  breaklinks=true,
  bookmarks=true,
  bookmarksnumbered=true,
  colorlinks=true,
  citecolor=black,
  filecolor=black,
  linkcolor=black,
  menucolor=black,
  urlcolor=black,
  pdftitle={An exemplary course},
  pdfauthor={Some professor, an assistant},
  pdfkeywords={An exemplary course}
]{hyperref}
\urlstyle{sf}

% --- Voreinstellungen für Name des Tutoriums etc. ---
\author{Some professor, an assistant}
\term{Summer~Semester~2018}
\course{An exemplary course}
\date{18.04.2342}
\topic{Introductory Exercises}
\sheet{1}
\printsolutions % comment this out if solutions shall not be printed to the PDF

\begin{document}

\begin{exercise}
  Let $\Sigma\coloneqq\{a,b,c\}$ and define
  \begin{alignat*}{1}
    \mathcal{L}(\Sigma):=\int f\mathrm{d}\mu.
  \end{alignat*}
  We further test some formulae as follows.

  $C=A\sqcap B\sqcap\exists r.A\sqcap\forall r.(B\sqcap\exists s.A)$

  Man kann auch deutschen Text schreiben, nämlich indem die Option \texttt{german} für die Klasse \texttt{tudscrexsht} gewählt wird.

  ä Ä ö Ö ü Ü ß
\end{exercise}

\begin{hint*}
  Use your brain.
\end{hint*}

\begin{hint}
  This is too easy!
\end{hint}

\begin{exercise*}
  This is a hard exercise...
\end{exercise*}

\begin{solution}
  TBA
\end{solution}

\end{document}
